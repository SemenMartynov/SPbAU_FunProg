\documentclass[a4paper,12pt]{article} %размер бумаги устанавливаем А4, шрифт 12пунктов

\usepackage[utf8]{inputenc}%включаем свою кодировку: koi8-r или utf8 в UNIX, cp1251 в Windows
\usepackage[english,russian]{babel}%используем русский и английский языки с переносами
\usepackage{amsmath} %подключаем нужные пакеты расширений 
\usepackage{color} %цвета
\usepackage{bm} % шрифты для красивых буковок =)

\usepackage{geometry} % Меняем поля страницы
\geometry{left=2cm} % левое поле
\geometry{right=1.5cm} % правое поле
\geometry{top=1.5cm} % верхнее поле
\geometry{bottom=1.5cm} % нижнее поле

\setcounter{tocdepth}{1}% chapter и section;

\begin{document}

\begin{flushright}
Практика 2. Рекурсия и редукция.

Мартынов Семён

\hrulefill
\end{flushright}

\section{Домашнее задание}

\begin{enumerate}

{\item Приведите пример замкнутого чистого $\lambda$-терма находящегося
	\begin{itemize}% начало помеченного списка
		{\item в слабой головной нормальной форме, но не в головной нормальной форме;}
		{\item в головной нормальной форме, но не в нормальной форме.}
	\end{itemize}% конец помеченного списка
}

Уточнить.





% -----------------------------------------------------------------------------
{\item Постройте функцию tail, возвращающую хвост списка, например}

zp $\equiv$ pair 0 0 \\
sp $\equiv$ $\lambda$ xp.pair(snd p)(cons x (snd p))

Будем двигаться по списку, складывая его во второй элемент, а в первом будет сам хвост. Останется только забрать этот первый элемент.

tail = $\lambda$ m.fst (m sp zp)

% -----------------------------------------------------------------------------
{\item Постройте функцию sum, суммирующую элементы списка, например
	\begin{center}
		$\mathrm{sum \ [5,3,2] = 10}$
	\end{center}
}

sum = $\lambda$ m.iif(empty m) m plus (head m)(sum(tail m))


% -----------------------------------------------------------------------------
{\item Используя $\bm{\mathrm{Y}}$-комбинатор, сконструируйте}
	\begin{itemize}% начало помеченного списка
		{\item «пожиратель», то есть такой терм $\mathrm{F}$, который для любого $\mathrm{M}$ обеспечивает $\mathrm{F \ M = F}$.}

			FM = F \\
			F = $\lambda$ z.F \\
			F = ($\lambda$ xz.x)F \\
			F = $\bm{\mathrm{Y}}$ ($\lambda$ xz.x)

		{\item терм $\mathrm{F}$ таким образом, чтобы для любого $\mathrm{M}$ выполнялось $\mathrm{F \ M = M \ F}$.}

			FM = MF \\
			F = $\lambda$ z.z F \\
			F = ($\lambda$ xz.zx) F \\
			F = $\bm{\mathrm{Y}}$ ($\lambda$ xz.zx)

		{\item терм $\mathrm{F}$ таким образом, чтобы для любых термов $\mathrm{M}$ и $\mathrm{N}$ выполнялось $\mathrm{F \ M \ N = N \ F \ (N \ M \ F)}$.}

			FMN = NF(NMF) \\
			FM = $\lambda$ z.zF(zMF) \\
			F = $\lambda$ xz.zF(zxF) \\
			F = ($\lambda$ yxz.zy(zxy)) F \\
			F = $\bm{\mathrm{Y}}$ ($\lambda$ yxz.zy(zxy))

	\end{itemize}% конец помеченного списка

% -----------------------------------------------------------------------------
{\item Пусть имеются взаимно-рекурсивное определение функций $\mathrm{f}$ и $\mathrm{g}$:
	\begin{center}
		$\mathrm{f = F \ f \ g}$ \\
		$\mathrm{g = G \ f \ g}$
	\end{center}
Используя Y-комбинатор, найдите нерекурсивные определения для $\mathrm{f}$ и $\mathrm{g}$.
}

	Введём P = pair f g, тогда f = fst P и g = snd P\\
	Выполним подстановку\\
	P = pair (F(fst P)(snd P))(G(fst P)(snd P)) \\
	P = (pair (F(fst P)(snd P))(G(fst P)(snd P)))P = AP \\
	P = YA $\Rightarrow$ f = fst YA, g = snd YA

\end{enumerate}

\end{document}
%
% Логические значения
%
% TRUE  = λx y. x
% FALSE = λx y. y
% IIF = λe x y. e x y
% AND = λx y. x y FALSE 
% 
% Пары
% 
% PAIR = λx y f. f x y
% FST p = p TRUE
% SND p = p FALSE
% 
% Числа
% 
% 0 = λf x. x
% SUC = λn f x. n f (f x)
% ISZERO = λn. (λx. FALSE) TRUE
% PLUS = λ m n f x. m f (n f x)
% MULT = λ m n f x. m (n f) x
% ZZ = PAIR 0 0
% SS = λp. PAIR (SND p) (PLUS (SUC 0) (SND p))
% PRD = λm. FST (m SS ZZ)
% SUBSTR = λm n. n PRD m
% GE = λm n. ISZERO (m PRD n)
% LE = λm n. ISZERO (n PRD m)
% EQ = λm n. AND (GE m n) (LE m n)
% MAX = λm n. IIF (GE m n) m n
% 
% Списки
% 
% NIL = λc n. n
% CONS = λx l c n. c x (l c n)
% LENGTH = λl. l(λx. SUC) 0
% HEAD = λl. l(λx y. x) NIL
% TAIL = λl. SND (l(λx p. PAIR (CONS x (FST p)) (FST p)) (PAIR NIL NIL))
% 
% Рекурсия
% 
% FIX = λf. (λx. f (λy. x x y)) (λx. f (λy. x x y))
% 
% Ну и, наконец,
% 
% F = λl. F1 l (LENGTH l) 0
% F1 = FIX F2
% F2 = λf l n i. IIF (EQ n i) 0 (MAX (PLUS( MULT (HEAD l) n) (SUC i)) (f (TAIL l) n (SUC i)))
