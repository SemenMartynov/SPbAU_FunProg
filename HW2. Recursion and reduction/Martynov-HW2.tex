\documentclass[a4paper,12pt]{article} %размер бумаги устанавливаем А4, шрифт 12пунктов
\usepackage[T1,T2A]{fontenc}
\usepackage[utf8]{inputenc}%включаем свою кодировку: koi8-r или utf8 в UNIX, cp1251 в Windows
\usepackage[english,russian]{babel}%используем русский и английский языки с переносами
\usepackage{amsmath} %подключаем нужные пакеты расширений
\usepackage{changepage} %пакет для отступов
\usepackage{color} %цвета

\usepackage{geometry} % Меняем поля страницы
\geometry{left=2cm}% левое поле
\geometry{right=1.5cm}% правое поле
\geometry{top=1.5cm}% верхнее поле
\geometry{bottom=1.5cm}% нижнее поле

\setcounter{tocdepth}{1}% chapter и section;

\begin{document}

\begin{flushright}
{\color{red} 24 фев 2014}

Практика 2. Рекурсия и редукция

Мартынов Семён

\hrulefill
\end{flushright}

\section{Домашнее задание}

\begin{enumerate}

{\item Приведите пример замкнутого чистого $\lambda$-терма находящегося}
\begin{itemize}
{\item в слабой головной нормальной форме, но не в головной нормальной форме;}
{\color{blue}
	\begin{adjustwidth}{1em}{0pt}
	%%%%%%%%%%%%%%%%%%% РЕШЕНИЕ %%%%%%%%%%%%%%%%%%%
	\end{adjustwidth}
}
{\item в головной нормальной форме, но не в нормальной форме.}
{\color{blue}
	\begin{adjustwidth}{1em}{0pt}
	%%%%%%%%%%%%%%%%%%% РЕШЕНИЕ %%%%%%%%%%%%%%%%%%%
	\end{adjustwidth}
}
\end{itemize}

% -----------------------------------------------------------------------------

{\item Напишите функции: minus, вычитающую числа Чёрча, equals, сравнивающую два числа Чёрча на предмет равенства, а также всевозможные неравенства, строгие и нестрогие, lt, gt, le, ge.}
{\color{blue}
	\begin{adjustwidth}{1em}{0pt}
	%%%%%%%%%%%%%%%%%%% РЕШЕНИЕ %%%%%%%%%%%%%%%%%%%
	\end{adjustwidth}
}
  
% -----------------------------------------------------------------------------

{\item Постройте функции:}
\begin{itemize}
{\item sum суммирующую элементы списка, например
   	\begin{center}
	$\mathrm{sum \ [5,3,2] = 10}$
\end{center}
}

{\color{blue}
	\begin{adjustwidth}{1em}{0pt}
    % Временно решение
	sum = $\lambda$ m.iif(empty m) m plus (head m)(sum(tail m))
	\end{adjustwidth}
}


{\item length вычисляющую длину списка, например
	\begin{center}
	$\mathrm{length \ [5,3,2] = 3}$
\end{center}
}

{\color{blue}
	\begin{adjustwidth}{1em}{0pt}
	%%%%%%%%%%%%%%%%%%% РЕШЕНИЕ %%%%%%%%%%%%%%%%%%%
	\end{adjustwidth}
}

\end{itemize}

% -----------------------------------------------------------------------------

{\item Постройте функцию tail, возвращающую хвост списка, например
	\begin{center}
		$\mathrm{tail \ [5,3,2] = [3,2]}$
	\end{center}
}

{\color{blue}
	\begin{adjustwidth}{1em}{0pt}
	zp $\equiv$ pair nil nil \\
	sp $\equiv$ $\lambda$ xp.pair(snd p)(cons x (snd p))

	Будем двигаться по списку, складывая его во второй элемент, а в первом будет сам хвост. Останется только забрать этот первый элемент.

	tail = $\lambda$ m.fst (m sp zp)
	\end{adjustwidth}
}

% -----------------------------------------------------------------------------

{\item Используя $\mathrm{Y}$-комбинатор, сконструируйте}
\begin{itemize}
{\item «пожиратель», то есть такой терм $\mathrm{F}$, который для любого $\mathrm{M}$ обеспечивает $\mathrm{F \ M = F}$.}

{\color{blue}
	\begin{adjustwidth}{1em}{0pt}
	FM = F \\
	F = $\lambda$ z.F \\
	F = ($\lambda$ xz.x)F \\
	F = $\mathrm{Y}$ ($\lambda$ xz.x)
	\end{adjustwidth}
}    

{\item терм $\mathrm{F}$ таким образом, чтобы для любого $\mathrm{M}$ выполнялось $\mathrm{F \ M = M \ F}$.}

{\color{blue}
	\begin{adjustwidth}{1em}{0pt}
	FM = MF \\
	F = $\lambda$ z.z F \\
	F = ($\lambda$ xz.zx) F \\
	F = $\mathrm{Y}$ ($\lambda$ xz.zx)
	\end{adjustwidth}
}

{\item терм $\mathrm{F}$ таким образом, чтобы для любых термов $\mathrm{M}$ и $\mathrm{N}$ выполнялось $\mathrm{F \ M \ N = N \ F \ (N \ M \ F)}$.}


{\color{blue}
	\begin{adjustwidth}{1em}{0pt}
	FMN = NF(NMF) \\
	FM = $\lambda$ z.zF(zMF) \\
	F = $\lambda$ xz.zF(zxF) \\
	F = ($\lambda$ yxz.zy(zxy)) F \\
	F = $\mathrm{Y}$ ($\lambda$ yxz.zy(zxy))
	\end{adjustwidth}
}

\end{itemize}

% -----------------------------------------------------------------------------

{\item Пусть имеются взаимно-рекурсивное определение функций $\mathrm{f}$ и $\mathrm{g}$:
	\begin{center}
		$\mathrm{f = F \ f \ g}$ \\
		$\mathrm{g = G \ f \ g}$
	\end{center}
	Используя Y-комбинатор, найдите нерекурсивные определения для $\mathrm{f}$ и $\mathrm{g}$.
}

{\color{blue}
	\begin{adjustwidth}{1em}{0pt}
	Введём P = pair f g, тогда f = fst P и g = snd P.
    
	Выполним подстановку
    
	P = pair (F(fst P)(snd P))(G(fst P)(snd P)) \\
	P = (pair (F(fst P)(snd P))(G(fst P)(snd P)))P = AP \\
	P = YA $\Rightarrow$ f = fst YA, g = snd YA
	\end{adjustwidth}
}

\end{enumerate}

\end{document}
