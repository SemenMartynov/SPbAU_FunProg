\documentclass[a4paper,12pt]{article} %размер бумаги устанавливаем А4, шрифт 12пунктов
\usepackage[utf8]{inputenc}%включаем свою кодировку: koi8-r или utf8 в UNIX, cp1251 в Windows
\usepackage[english,russian]{babel}%используем русский и английский языки с переносами
\usepackage{amsmath} %подключаем нужные пакеты расширений
\usepackage{changepage} %пакет для отступов
\usepackage{color} %цвета

\usepackage{geometry} % Меняем поля страницы
\geometry{left=2cm}% левое поле
\geometry{right=1.5cm}% правое поле
\geometry{top=1.5cm}% верхнее поле
\geometry{bottom=1.5cm}% нижнее поле

\setcounter{tocdepth}{1}% chapter и section;

\begin{document}

\begin{flushright}
{\color{red} 17 фев 2014 (с доп. от 22 фев 2014)}

Практика 1. Чистое лямбда-исчисление как язык программирования

Мартынов Семён

\hrulefill
\end{flushright}

\section{Задачи, пропущенные в ходе практики}

\begin{enumerate}
{\item Проверьте, что ожидаемые свойства iszro выполняются.}

iszro $\equiv$ $\lambda$ n. n ($\lambda$ x. fls) tru

  \begin{adjustwidth}{1em}{0pt}
  Для проверки рассмотрим два случая.
  
  iszro 0 $\equiv$ \\
  ($\lambda$ n. n ($\lambda$ x. fls) tru) $\lambda$ s z. z $\equiv$ \\
  ($\lambda$ s z. z) ($\lambda$ x. fls) tru $\equiv$ \\
  ($\lambda$ z. z) tru $\equiv$ \\
  tru
  
  iszro 1 $\equiv$ \\
  ($\lambda$ n. n ($\lambda$ x. fls) tru) $\lambda$ s z. s z $\equiv$ \\
  ($\lambda$ s z. s z) ($\lambda$ x. fls) tru) $\equiv$ \\
  ($\lambda$ z. ($\lambda$ x. fls) z) tru $\equiv$ \\
  ($\lambda$ x. fls) tru $\equiv$ \\
  fls.
  \end{adjustwidth}

\end{enumerate}

\section{Задачи, не решённые во время практики}

\begin{enumerate}

{\item Попробуйте найти определение plus с использованием succ.}

\begin{adjustwidth}{1em}{0pt}
plus $\equiv$ $\lambda$m n. m succ n
    \end{adjustwidth}

{\item Проверьте, что ожидаемые свойства умножения выполняются.}

mult1 $\equiv$ $\lambda$m n. m (plus n) 0

mult2 $\equiv$ $\lambda$m n s z. m (n s) z
    
\begin{adjustwidth}{1em}{0pt}
mult1 2 3 $\equiv$ \\
    ($\lambda$m n. m (plus n) 0) 2 3 $\equiv$ \\
    2 (plus 3) 0 $\equiv$ \\
    $\lambda$s z. s( s z) (plus 3) 0 $\equiv$ \\
    (plus 3) ((plus 3) 0) $\equiv$ \\
    6
    
    mult2 2 3 $\equiv$ \\
    ($\lambda$m n s z. m (n s) z) 2 3 $\equiv$ \\
    $\lambda$s z. 2 (3 s) z $\equiv$ \\
    $\lambda$s z. 2 (($\lambda$s z. s (s (s z))) s) z $\equiv$ \\
    $\lambda$s z. 2 ($\lambda$z. s (s (s z))) z $\equiv$ \\
    $\lambda$s z. ($\lambda$s z. s (s z)) ($\lambda$z. s (s (s z))) z $\equiv$ \\
    $\lambda$s z. ($\lambda$z. ($\lambda$z. s (s (s z))) (($\lambda$z. s (s (s z))) z)) z $\equiv$ \\
    $\lambda$s z. ($\lambda$z. ($\lambda$z. s (s (s z))) (s (s (s z)))) z $\equiv$ \\
    $\lambda$s z. ($\lambda$z. (s (s (s (s (s (s z))))))) z $\equiv$ \\
    $\lambda$s z. s (s (s (s (s (s z))))) $\equiv$ \\
    6
    \end{adjustwidth}

{\item Можно ли mult2 записать короче?}

\begin{adjustwidth}{1em}{0pt}
mult2 $\equiv$ $\lambda$m n s. m (n s)
    \end{adjustwidth}

\end{enumerate}

\section{Домашнее задание}

\begin{enumerate}
{\item Выполните подстановку}

\begin{enumerate}
{\item $\lambda$ y z. x y w (z x) $\quad$ [x := $\lambda$ y. y w]}

$\lambda$ y z. ($\lambda$ y. y w) y w (z ($\lambda$ y. y w)) $\equiv$ $\lambda$ y z. y w w (z ($\lambda$ y. y w))

{\item $\lambda$ x y. x y ($\lambda$ x. x y) x $\quad$ [x := $\lambda$ z. z]}

$\lambda$ x y. x y ($\lambda$ x. x y) x

{\item x y ($\lambda$ x z. x y z) y $\quad$ [y := x z]}

x (x z) ($\lambda$ x$^\prime$ z$^\prime$. x$^\prime$ (x z) z$^\prime$) (x z)

\end{enumerate}

{\item Уберите лишние скобки и при возможности выполните $\beta$-преобразование}

\begin{enumerate}

{\item (x ($\lambda$ x.((x y) x)) y)}

{\color{red}(}x ($\lambda$ x.((x y) x)) y{\color{red})} $\equiv$ \\
x ($\lambda$ x.({\color{red}(}x y{\color{red})} x)) y $\equiv$ \\
x ($\lambda$ x.{\color{red}(}x y x{\color{red})}) y $\equiv$ \\
x ($\lambda$ x. x y x) y

{\item (($\lambda$ p.($\lambda$ q.((q (p r)) s))) ((q (p r)) s))}

{\color{red}(}($\lambda$ p.($\lambda$ q.((q (p r)) s))) ((q (p r)) s){\color{red})} $\equiv$ \\
($\lambda$ p.($\lambda$ q.({\color{red}(}q (p r){\color{red})} s))) ((q (p r)) s) $\equiv$ \\
($\lambda$ p.($\lambda$ q.{\color{red}(}q (p r) s{\color{red})})) ((q (p r)) s) $\equiv$ \\
($\lambda$ p.{\color{red}(}$\lambda$ q. q (p r) s{\color{red})}) ((q (p r)) s) $\equiv$ \\
($\lambda$ p q. q (p r) s) ({\color{red}(}q (p r){\color{red})} s) $\equiv$ \\
($\lambda$ p q. q (p r) s) (q (p r) s) $\equiv$ \\
$\lambda$ q$^\prime$. q$^\prime$ ({\color{red}(}q (p r) s{\color{red})} r) s = \\
$\lambda$ q$^\prime$. q$^\prime$ (q (p r) s r) s.

\end{enumerate}

{\item Покажите, что для любых M и N выполняется $\lambda$ x. M N = \textbf{S} ($\lambda$ x. M) ($\lambda$ x. N)}

    \begin{adjustwidth}{1em}{0pt}
    \textbf{S} ($\lambda$ x. M) ($\lambda$ x. N) $\equiv$ \\
    ($\lambda$ f g x$^\prime$ . f x$^\prime$ ( g x$^\prime$ ) ) ($\lambda$ x. M) ($\lambda$ x. N) $\equiv$ \\
    $\lambda$x$^\prime$. ($\lambda$ x. M) x$^\prime$ (($\lambda$ x. N) x$^\prime$) $\equiv$ \\
    $\lambda$x$^\prime$. ($\lambda$ x. M) x$^\prime$ (N) $\equiv$ \\
    $\lambda$ x$^\prime$. M N $\equiv$ \\
    $\lambda$ x. M N
    \end{adjustwidth}

%\newpage

{\item Покажите, что}

\begin{enumerate}

{\item \textbf{SKK = I}}

    \textbf{SKK} $\equiv$ \\
    $\lambda$x. \textbf{K} x (\textbf{K} x) $\equiv$ \\
    $\lambda$x. x $\equiv$ \\
    \textbf{I}


{\item \textbf{B = S (K S) K}}

    \textbf{S (K S) K} = \\
    $\lambda$x. (\textbf{K S}) x (\textbf{K} x) = \\
    $\lambda$x. \textbf{K S} x (\textbf{K} x) = \\
    $\lambda$x. \textbf{S} (\textbf{K} x) = \\
    $\lambda$x$^\prime$. \textbf{S} (\textbf{K} x$^\prime$) = \\
    $\lambda$x$^\prime$. ($\lambda$f g x. f x (g x)) (\textbf{K} x$^\prime$) = \\
    $\lambda$x$^\prime$. ($\lambda$g x. (\textbf{K} x$^\prime$) x (g x)) = \\
    $\lambda$x$^\prime$. ($\lambda$g x. \textbf{K} x$^\prime$ x (g x)) = \\
    $\lambda$x$^\prime$. ($\lambda$g x. x$^\prime$ (g x)) = \\
    $\lambda$x$^\prime$ g x. x$^\prime$ (g x) = \\
    $\lambda$f g x. f (g x) = \\
\textbf{B}.

\end{enumerate}

{\item Реализуйте функцию возведения в степень для чисел Чёрча.}

    \begin{adjustwidth}{1em}{0pt}
    pow $\equiv$ $\lambda$m n. n (mult m) 1
    
    либо
    
    pow $\equiv$ $\lambda$m n . n m
    \end{adjustwidth}

\end{enumerate}

\end{document}
